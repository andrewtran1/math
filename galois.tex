\documentclass[12pt]{article}

\usepackage{amsmath}
\usepackage{amsthm}
\usepackage{mathrsfs}

\newtheorem{defn}{Definition}
\newtheorem{lem}{Lemma}
\newtheorem{thm}{Theorem}

\title{Galois Theory}
\author{Andrew Tran}
\date{}

\begin{document}
\maketitle

\section{Polynomial Rings}
\subsection{Adjuctions and their Properties}
\begin{defn}
  Let $\mathscr{C}$  and $\mathscr{D}$ be categories. Then an adjunction from $\mathscr{C}$ to $\mathscr{D}$ is a $4$-tuple $(\eta,\epsilon):F\vdash G$, where $F:\mathscr{C}\to\mathscr{D}$ and $G:\mathscr{D}\to\mathscr{C}$ are functors, $\eta:1_{\mathscr{C}}\Rightarrow GF$ and $\epsilon:FG\Rightarrow1_{\mathscr{D}}$ are natural transformations and the following conditions hold:
  \begin{enumerate}
    \item For every $X \in Ob\mathscr{C}$, $\epsilon_{FX}\circ F\eta_{X} = 1_{FX}$ 
    \item For every $Y \in Ob\mathscr{D}$, $G\epsilon_{Y}\circ\eta_{GY} = 1_{GY}$
  \end{enumerate}
\end{defn}

\subsection{Polynomial Ring Adjoints}
\begin{defn}
  Let $CRing$ be the category of commutative rings and ring homomorphisms and $CRing^{*}$ be the category of pointed commutative rings and pointed ring homomorphisms. Then the functor $$U:CRing^{*}\to CRing:[\phi:(R,a)\to(S,b)]\mapsto[\phi:R\to S]$$ is called the forgetful functor.
\end{defn}

\begin{defn}
  A polynomial ring adjoint is a triple $(\Pi,\eta,\epsilon)$ such that $$(\eta,\epsilon):\Pi\vdash U$$ is an adjunction from $CRing$ to $CRing^{*}$.
\end{defn}

\noindent For the remainder of these notes, let $(\Pi,\eta,\epsilon)$ be a fixed polynomial ring adjoint. We define $\pi = U\Pi$. 

\begin{defn} 
  Let $R$ be a commutative ring and let $X$ be the distinguished point of the pointed ring $\Pi R$. Then $\Pi R$ is referred to as the polynomial ring over $R$ with indeterminate element $X$. The elements $p \in \Pi R$ are referred to as (formal, univariate) polynomials over $R$.
\end{defn}

\begin{defn}
  Let $(R,a)$ be a pointed commutative ring. Then we define $eval_{(R,a)}=U\epsilon_{(R,a)}:\pi R \to R$ and refer to $eval_{(R,a)}$ as the evaluation homomorphism on $R$ at $a$.
\end{defn}

\begin{defn}
  Let $R$ be a commutative ring. Then we refer to $\eta_{R}:R\to\pi R$ as the $R$ embedding homomorphism.
\end{defn}

\begin{thm}
  For any pointed commutative ring $(R,a)$, $\eta_{R}:R\to\pi R$ is injective and $eval_{(R,a)}:\pi R\to R$ is surjective.
\end{thm}

\begin{proof}
  Let $(R,a)$ be a pointed commutative ring. By definition of adjunctions, $$eval_{(R,a)}\circ\eta_{R} = U\epsilon_{(R,a)}\circ\eta_{U(R,a)} = 1_{U(R,a)}$$ In other words, $eval_{(R,a)}$ is a right invertible function and $\eta_{R}$ is a left invertible function. This implies that they are surjective and injective, respectively.
\end{proof}

\begin{defn}
  Let $R$ be a commutative ring. We define the subring of constant polynomials over $R$, $\overline{R}\subseteq\pi R$, to be the subring of $\pi R$ given by the image of $R$ under the embedding homomorphism $\eta_R$.
\end{defn}

\begin{lem}
  For any commutative ring $R$, $R \cong \overline{R}$.
\end{lem}

\begin{proof}
  It has already been proven that $\eta_R:R\to\overline{R}$ is injective and is, by construction, surjective on $\overline{R}$.
\end{proof}

\begin{thm}
  Let $R$ be a commutative ring. Suppose $X\in\pi R$ is the indeterminate element. Then the following are equivalent:
  \begin{enumerate}
    \item $X\in\overline{R}$
    \item $X$ is a unit in $\pi R$
    \item $R$ is a zero ring
    \item $\pi R$ is a zero ring
  \end{enumerate}
\end{thm}

\begin{proof}
  Let $R$ be a commutative ring and $X\in\pi R$ be the indeterminate element.
  \begin{enumerate}
    \item Suppose that $X\in\overline{R}$. Let $x=\eta^{-1}_{R}(X)\in R$.
      By properties of adjunctions, the pair $(\eta_R, (\pi R, X))$ is an initial morphism from $R$ to $U$. Additionally, $$\eta_R:R\to U(\pi R, 1_{\pi R})$$
      So, from the definition of initial morphisms, there is a unique pointed ring homomorphism $\phi:(\pi R,X)\to(\pi R,1_{\pi R})$
      such that $$U\phi\circ\eta_{R} = \eta_{R}$$
      Therefore,
      $$X = \eta_{R}(x) = U\phi\circ\eta_{R}(x) = \phi(X) = 1_{\pi R}$$
      So, $X$ is a unit in $\pi R$.
    \item Suppose that $X$ is a unit in $\pi R$. Let $Y\in\pi R$ such that $XY = 1_{\pi R}$. Then
      \begin{align*}
        eval_{(R,0_R)}(XY) &= eval_{(R,0_{R})}(1_{\pi R})\\ 
        eval_{(R,0_R)}(X)\cdot eval_{(R,0_R)}(Y) &= 1_{R}\\ 
        0_{R}\cdot eval_{(R,0_R)}(Y) &= 1_{R}\\
        0_{R}&= 1_{R} 
      \end{align*}
      So, $R$ is a zero ring.
    \item Suppose that $R$ is a zero ring. Then $1_R = 0_R$. Therefore, $$\eta_{R}(1_R)=\eta_{R}(0_R)$$
      Since $\eta_{R}$ is a ring homomorphism,
      $$1_{\pi R} = \eta_{R}(1_R) = \eta_{R}(0_R) = 0_{\pi R}$$
      Thus, $\pi R$ is a zero ring.
    \item Suppose $\pi R$ is a zero ring. Then $X=0_{\pi R}=\eta_{R}(0_{R})\in\overline{R}$. 
  \end{enumerate}
  Therefore, these conditions are all equivalent.
\end{proof}

\begin{thm}
  Let $R$ be a commutative ring. Then the indeterminate element $X\in\pi R$ is not a zero divisor in $\pi R$.
\end{thm}

\end{document}

\documentclass[12pt]{article}

\usepackage{amsmath}
\usepackage{amsthm}
\usepackage{amsfonts}
\usepackage{mathrsfs}

\newtheorem{defn}{Definition}
\newtheorem{lem}{Lemma}
\newtheorem{thm}{Theorem}

\title{Galois Theory}
\author{Andrew Tran}
\date{}

\begin{document}
\maketitle

\section{Polynomial Rings}
\subsection{Adjuctions and their Properties}
\begin{defn}
  Let $\mathscr{C}$  and $\mathscr{D}$ be categories. Then an adjunction from $\mathscr{C}$ to $\mathscr{D}$ is a $4$-tuple $(\eta,\epsilon):F\vdash G$, where $F:\mathscr{C}\to\mathscr{D}$ and $G:\mathscr{D}\to\mathscr{C}$ are functors, $\eta:1_{\mathscr{C}}\Rightarrow GF$ and $\epsilon:FG\Rightarrow1_{\mathscr{D}}$ are natural transformations and the following conditions hold:
  \begin{enumerate}
    \item For every $X \in Ob\mathscr{C}$, $\epsilon_{FX}\circ F\eta_{X} = 1_{FX}$ 
    \item For every $Y \in Ob\mathscr{D}$, $G\epsilon_{Y}\circ\eta_{GY} = 1_{GY}$
  \end{enumerate}
\end{defn}

\subsection{Polynomial Ring Adjoints}
\begin{defn}
  Let $CRing$ be the category of commutative rings and ring homomorphisms and $CRing^{*}$ be the category of pointed commutative rings and pointed ring homomorphisms. Then the functor $$U:CRing^{*}\to CRing:[\phi:(R,a)\to(S,b)]\mapsto[\phi:R\to S]$$ is called the forgetful functor.
\end{defn}

\begin{defn}
  A polynomial ring adjoint is a triple $(\Pi,\eta,\epsilon)$ such that $$(\eta,\epsilon):\Pi\vdash U$$ is an adjunction from $CRing$ to $CRing^{*}$.
\end{defn}

\noindent For the remainder of these notes, let $(\Pi,\eta,\epsilon)$ be a fixed polynomial ring adjoint. We define $\pi = U\Pi$. 

\begin{defn} 
  Let $R$ be a commutative ring and let $x$ be the distinguished point of the pointed ring $\Pi R$. Then $\Pi R$ is referred to as the polynomial ring over $R$ with indeterminate element $x$. The elements $p \in \Pi R$ are referred to as (formal, univariate) polynomials over $R$.
\end{defn}

\begin{defn}
  Let $(R,a)$ be a pointed commutative ring. Then we define $eval_{(R,a)}=U\epsilon_{(R,a)}:\pi R \to R$ and refer to $eval_{(R,a)}$ as the evaluation homomorphism on $R$ at $a$.
\end{defn}

\begin{defn}
  Let $R$ be a commutative ring. Then we refer to $\eta_{R}:R\to\pi R$ as the $R$ embedding homomorphism.
\end{defn}

\begin{thm}
  For any pointed commutative ring $(R,a)$, $\eta_{R}:R\to\pi R$ is injective and $eval_{(R,a)}:\pi R\to R$ is surjective.
\end{thm}

\begin{proof}
  Let $(R,a)$ be a pointed commutative ring. By definition of adjunctions, $$eval_{(R,a)}\circ\eta_{R} = U\epsilon_{(R,a)}\circ\eta_{U(R,a)} = 1_{U(R,a)}$$ In other words, $eval_{(R,a)}$ is a right invertible function and $\eta_{R}$ is a left invertible function. This implies that they are surjective and injective, respectively.
\end{proof}

\begin{defn}
  Let $R$ be a commutative ring. We define the subring of constant polynomials over $R$, $\overline{R}\subseteq\pi R$, to be the subring of $\pi R$ given by the image of $R$ under the embedding homomorphism $\eta_R$.
\end{defn}

\begin{lem}
  For any commutative ring $R$, $R \cong \overline{R}$.
\end{lem}

\begin{proof}
  It has already been proven that $\eta_R:R\to\overline{R}$ is injective and is, by construction, surjective on $\overline{R}$.
\end{proof}

\begin{thm}
  Let $R$ be a commutative ring. Suppose $x\in\pi R$ is the indeterminate element. Then the following are equivalent:
  \begin{enumerate}
    \item $x$ is a unit in $\pi R$
    \item $x\in\overline{R}$
    \item $R$ is a zero ring
    \item $\pi R$ is a zero ring
  \end{enumerate}
\end{thm}

\begin{thm}
  Let $R$ be a commutative ring. Then the indeterminate element $x\in\pi R$ is not a zero divisor in $\pi R$.
\end{thm}

\begin{thm}
  Let $R$ be a non-trivial commutative ring and $x\in\pi R$ be the indeterminate element. Then for any polynomial $p\in\pi R$, $xp \neq 1_{\pi R}$.  
\end{thm}
\end{document}
